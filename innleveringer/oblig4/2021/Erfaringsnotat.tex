\documentclass[a4paper,10pt]{article}
\usepackage[utf8]{inputenc}
\usepackage[norsk]{babel}
% Standard stuff
\usepackage{amsmath,graphicx,varioref,verbatim,amsfonts,geometry}
% colors in text
\usepackage[usenames,dvipsnames,svgnames,table]{xcolor}
% Hyper refs
\usepackage[colorlinks]{hyperref}
\usepackage{caption}

% Document formatting
\setlength{\parindent}{0mm}
\setlength{\parskip}{1.5mm}

%Color scheme for listings
\usepackage{textcomp}
\definecolor{listinggray}{gray}{0.9}
\definecolor{lbcolor}{rgb}{0.9,0.9,0.9}

%Listings configuration
\usepackage{listings}
%Hvis du bruker noe annet enn python, endre det her for å få riktig highlighting.
\lstset{
	backgroundcolor=\color{lbcolor},
	tabsize=4,
	rulecolor=,
	language=python,
        basicstyle=\scriptsize,
        upquote=true,
        aboveskip={1.5\baselineskip},
        columns=fixed,
	numbers=left,
        showstringspaces=false,
        extendedchars=true,
        breaklines=true,
        prebreak = \raisebox{0ex}[0ex][0ex]{\ensuremath{\hookleftarrow}},
        frame=single,
        showtabs=false,
        showspaces=false,
        showstringspaces=false,
        identifierstyle=\ttfamily,
        keywordstyle=\color[rgb]{0,0,1},
        commentstyle=\color[rgb]{0.133,0.545,0.133},
        stringstyle=\color[rgb]{0.627,0.126,0.941}
        }
        
\newcounter{subproject}
\renewcommand{\thesubproject}{\alph{subproject}}
\newenvironment{subproj}{
\begin{description}
\item[\refstepcounter{subproject}(\thesubproject)]
}{\end{description}}

%Lettering instead of numbering in different layers
%\renewcommand{\labelenumi}{\alph{enumi}}
%\renewcommand{\thesubsection}{\alph{subsection}}

%opening
\title{IN1010 - Oblig 4 \\ Erfaringsnotat}
\author{Nils Taugbøl}

\begin{document}
\maketitle

Oppgaven så først skremmende ut, men det skyldes mye tekst.
Etterhvert som man kommer i gang og tar én deloppgave om gangen så blir det ganske greit.
Hvis man tar litt tid til å tenke over de mer utfordrende deloppgavene (spesielt hele E)
og danner seg et bilde av hva man ønsker blir det fort lettere å skrive kode siden man klarer letter å se for seg hvordan det skal bli.
Det var en omfattende oppgave og de tidligere klassene fikk noen endringer for å gjøre det lettere å bruke alle klassene som ønsket i Legesystem.
Hva som måtte settes inn i de allerede eksisterende klassene kom naturlig (f.eks. gi en id til Pasient-objektene), og implementasjonen av disse ekstra metodene og variablene var helt greit.

Den største fordelen med å jobbe alene er at det er mye mer effektiv kommunikasjon mellom hva som blir skrevet i koden og det man tenker.
Man mister ikke tid med prating, det er mye mer rett på sak og man har ingen å klage på/krangle med.
Derimot så må man gjøre alt selv, så det tar mye mer individuell arbeidstid.



\end{document}